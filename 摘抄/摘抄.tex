\documentclass[UTF8,oneside]{ctexbook}
\title{圣道的文录} 
\author{圣道} 
\date{\today} 
\usepackage{graphicx,amsmath,amssymb,fontmfizz}
\begin{document}
%正文内容 
\maketitle
\let\kaishu\relax
\tableofcontents
%懒人必备方法

\chapter{书籍摘录}
\section{《剑来》}
%-----------------------------------《剑来》
\mfApache \quad 自童年起,我便独自一人,照顾着历代星辰。

\mfApache \quad 任何一个真正的强者,应该以弱者的自由为边界。强者,向更强者出拳!

\mfApache \quad 风高浪快,万里骑乘蟾背,身游天阙,俯瞰积气濛濛。
醉里仙人摇桂树,人间换做清风。

\newpage
\section{《雪中悍刀行》}
%-----------------------------------《雪中悍刀行》
\mfApache \quad 人生但苦无妨,良人当归即好。

\mfApache \quad 天地是大,所站之地不过方寸地;人生苦短,
                才百年三万六千五百日,糊糊涂涂过一辈子就很好。

\mfApache \quad 敢问何为九天之云下垂,何为四海之水皆立?

\mfApache \quad 若不回头,谁替你救苦救难;如能转念,何须我大慈大悲.

\mfApache \quad 有人来时,入江湖,意气风发;
\par \quad \quad 去时,出江湖,问心无愧.

\mfApache \quad 浪成于微澜之间,风起于青萍之末。惊蛰一过,
                百虫群出,闻风而动。

\mfApache \quad 困难于易,为大于细。天下大事必作于细,天下难事必作于易。
                多易必多难,轻诺必寡信

\mfApache \quad 提笔前,云蒸霞蔚我去见圣贤仙佛;
提笔后,风清月白天地鬼神来拜我。

\mfApache \quad 所谓低头登山一甲子,方知昆仑上巅有盏灯。

\mfApache \quad 君不见北冥有鱼扶摇几万里,
君不见昆仑之颠仙人过仙门。
君不见男儿轻骑出凉裹尸还,
君不见女子红妆倚门到白首。

\mfApache \quad 一个人的快乐和苦难,因所居位置不同而不同,但深浅大致是相当的。
所以谁也不要瞧不起谁,谁都不要笑话谁,什么事情都能争取,
唯独从哪里投胎,却是这辈子如何用心用力也争取不来的,遇上不平事,
能认就是本事,能拼命就是了不起了。

\mfApache \quad 遇到什么事情,可想可不想的时候,多想一想,可做不可做的时候,
不妨去做一下。人活一世,自保无虞之际,只求自己念头通达,不顾他人的顺心如意,那样的
陆地神仙,不做也罢。

\mfApache \quad 人生天地间,当顶天立地,才算真逍遥。

\mfApache \quad 虽止步于立锥之地,神游却已千万里。不问我来自何处,且思我要去何方见谁。

\mfApache \quad 火绝烟沉右西极,谷静山空左北平。但使将军能百站战,不须天子筑长城。

\mfApache \quad 春秋一梦梦春秋。人活一世,不过就是生死两事,来时既哭,去时当笑。 

\mfApache \quad 人已死却不怨,未归人却不知。

\mfApache \quad 天道有常,不为圣贤而存,不为凶桀而亡。

\mfApache \quad 道教圣人有言,生死如睡,睡下可起,为生;睡后不可起,为死。
故而此间有大恐怖,人人生时不笑反哭,便是此理。佛典也云云息心得寂静,生死大恐怖。

\mfApache \quad 

\section{《龙族》}
%--------------------------------------------------------龙族
\mfApache \quad 可人不是断气的时候才真的死了。有人说人会死三次,
第一次是他断气的时候 ,从生物学上他死了;第二次是他下葬的时候,
人们来参加他的葬礼,怀念他的一生,然后在社会上他死了,不再会有他的位置;
第三次是最后一个记得他的人把他忘记的时候,那时候他才真正的死了。

\mfApache \quad 没有人会记得死的东西,所以要活下去,咬牙切齿的活下去!

\mfApache \quad 如果非要爱什么才能让你有信心活下去的话,不如爱我好了。

\mfApache \quad 在我可以吞噬这个世界之前,与其孤独跋涉,不如安然沉睡。

\mfApache \quad 有些路你和某人一起走,就长得离谱,你和另外一些人走,就短得让人舍不得迈开脚步。

\mfApache \quad 比孤独更可悲的事情,就是根本不知道自己很孤独,或者分明很孤独,却把自己都骗得相信自己不孤独。

\mfApache \quad 很多人都能轻易地说出宽恕二字,只是因为他们并不懂仇恨。

\mfApache \quad 你需要付出的,只是心底里那点小小的温软,从此坚硬如铁。

\mfApache \quad 胸腔的火焰,总有一天,为了某个人,会燃起来,焚天灭世。

\mfApache \quad 有些事你发狠就能牛逼,大部分事你怀着希望赌上命都没用。

\mfApache \quad 最孤单的人分两种,一种恨不得全世界都跟他一样倒霉,一种则希望别人能幸福,因为看到幸福的人,他也略略觉得温暖。

\mfApache \quad People come and go, we struggled with laughter and tears, 
and all the years have gone by,still Ihave you by my side. 
(你陪了我多少年,花开花落。一路上起起跌跌。)

\mfApache \quad 你觉得你为正义支付了代价,你觉得痛苦,因为你所遵从的正义并不是你自己心里真正想要的东西。你遵从的是别人教给你的“大义”,而不是你自己的心。

\mfApache \quad 如果神俯视世界,会凝视每个路人么?就像孩子蹲在树根旁看着忙忙碌碌进进出出的蚁群,拿着树根在蚁洞里捅来捅去,却不会真正凝视其中任何一只。

\mfApache \quad 他总是看着头顶唯一的方窗,渴望鸟儿一样飞翔,渴望什么东西从天而降改变他的人生。
\chapter{语句摘录}
\section{《超兽武装》}
\mfApache \quad 在我的面前,敌人不是逃之夭夭,就是一败涂地。

\mfApache \quad 天堂和地狱,没有我选择的权力,只有我被选择的命运。

\mfApache \quad 人的欲望,就像高山上的滚石一般,一旦开始,就再也停不下来了。

\mfApache \quad 这世上有两种人,一种人生命的目的,并不是为了存在,而是为了燃烧
,而另一种人却永远只有看着别人燃烧,让别人的光芒来照耀自己,究竟哪种人才是聪明人。

\mfApache \quad 当我到达高处,便发觉自己总是孤独的,无人同我说话,孤寂的严冬
令我发抖,我在高处究竟意欲何为。

\mfApache \quad 自从厌倦于追寻,我已学会一觅即中;自从一股逆风袭来,我已能抵御八面来风,驾舟而行。

\mfApache \quad 我的内心就像树一样,树越是向往高处的光亮,它的根就越要向下,向泥土,向黑暗的深处。

\mfApache \quad 已有的事,后必再有,已行的事,后必再行。

\mfApache \quad 当你经过七重的孤独,才能够成为真正的强者。我们的世界也由此而生。

\mfApache \quad 美,只不过是一瞬间的感觉,只有真实才是永恒的,而真实绝不会美!

\mfApache \quad 我们是来自黑暗深渊的灵魂,我们内心向往的是无限的黑暗,黑暗里没有爱。

\mfApache \quad 爱能创造一切,也能毁灭一切。当你用爱保护羊群不受狼的伤害,
那么对于狼,这种爱心就等于毁灭,因为他们会因此而活活饿死。这个世界本就如此,
不是狼死就是羊死,不是弱小的狼被饿死,就是弱小的羊被咬死。或许,
这世界太过残酷,然而,却因此而美丽。

\mfApache \quad 为了更强,我们才存在。

\mfApache \quad 我们飞翔的越高,在不能飞翔的人眼里,就越是渺小。

\mfApache \quad 为战而生,至死方休!
\section{《星游记》}
\mfApache \quad 给我高高地飞起来啊!

\mfApache \quad “不可能”这三个字,你说的太多了。

\mfApache \quad 所有人都会有害怕的东西。但是感到恐惧就躲起来,时间就会变得越来越小。
第一道闪光,第一束火苗,让人类进步的所有事物,都是由恐惧开始的。
把最恐怖的地方变成最美丽的景色,这才是男人的梦想。

\mfApache \quad 王冠会让戴上它的人高人一等,是地位的象征。
但是王冠的真正的意义是:当灾难从天而降的时候,我会为你抵挡一切,
永远只让你们看到金色的希望。

\mfApache \quad 真正相信奇迹的家伙,本身就和奇迹一样了不起啊!

\mfApache \quad 星空之所以美丽,就是因为在无限的宇宙中,不管黑暗如何蔓延,
都有星星的光芒把它照亮。世界也是这样,有绝望的地方,就会有希望产生。

\mfApache \quad 世界对弱者的无情,比任何盾牌都还要坚固。

\mfApache \quad 就算这样,也要笑着去努力啊。

\mfApache \quad 拥有希望的人,和满天的星星一样,是永远不会孤独的。
找到和自己一样的星星,把通往自由的路,照亮吧!

\mfApache \quad 就算是不能成为什么伟大的大人物,如果能尽自己的能力,为别人带来一点的幸福,应该也是很值得
努力的大事情了。为越多的人带来幸福,才能称为越大的梦想。

\mfApache \quad 放下武器,有时候比拿起武器还要难。

\mfApache \quad \quad \quad \quad \quad 《再飞行》

\quad \quad \quad \quad \quad 眼前重复的风景

\quad \quad \quad \quad \quad 渐渐模糊的约定

\quad \quad \quad \quad \quad 星空下流浪的你

\quad \quad \quad \quad \quad 仍然秘密的距离

\quad \quad \quad \quad \quad 温度消失的瞬间

\quad \quad \quad \quad \quad 无法触摸的明天

\quad \quad \quad \quad \quad 没有引力的世界

\quad \quad \quad \quad \quad 没有脚印的光年

\quad \quad \quad \quad \quad 还在等着你出现

\quad \quad \quad \quad \quad 日日夜夜自转的行星

\quad \quad \quad \quad \quad 到处遮满别人的背影

\quad \quad \quad \quad \quad 让风吹散混乱的呼吸

\quad \quad \quad \quad \quad 快快清醒 yeyeye

\quad \quad \quad \quad \quad 静静照亮原来的自己

\quad \quad \quad \quad \quad 天空撒满忽然的光明

\quad \quad \quad \quad \quad 眼中只要绚烂的天际

\quad \quad \quad \quad \quad 再飞行

\section{人这一辈子}
%-----------------------------------《人这一辈子》
\mfApache \quad 一个人一生中所下的雪,别人无法全部看见。

\mfApache \quad 所有人死后肯定去天堂;难道没有地狱吗?这不就是地狱吗?

\mfApache \quad 有人问,到底是什么样的终点,才配得上这一路的颠沛流离?人生百态
,其实每个人都在自己的生命里,孤独过冬。

\mfApache \quad 虽有遗憾,绝不后悔。

\mfApache \quad 一如既往,孤独相伴;万千纷扰,与我何干。

\mfApache \quad 当我年轻的时候,我想着可以改变世界。后来当我年纪渐长,我发现改变世界很难,也许
我可以改变我的国家,年纪越来越大,发现改变国家也很难,也许我可以改变我的家庭,当我行将就木,我发现如果
我从一开始就想着改变我自己,也许就能慢慢改变我的家庭,通过家庭的帮助说不定会影响我的国家,甚至说不定有一天
我能影响这个世界。

\mfApache \quad 每一个优秀的人,都有一段沉默的时光。那一段时光,是付出了很多努力,
忍受孤独和寂寞,不抱怨不诉苦。日后说起时,连自己都能被感动的日子。

\mfApache \quad 人只有知道自己的无知后,才能从骨子里谦和起来,不再恃才傲物,
不再咄咄逼人。所以说,人总是越活越平和,我们称之为成长,成长就是慢慢的尊重自己一样尊重他人,
承认自己的无知不等于否定自己,而是为了改善自己。

\mfApache \quad 人一旦堕落,哪怕是短短的几年,上帝就会以最快的速度,收走你的
天赋和力量。

\mfApache \quad 你所做的事情,也许暂时看不到成果,但不要会心或者焦虑,
你不是没有成长,而是在扎根。

\mfApache \quad 二十几岁的你,迷茫又着急。你想要房子,车子,你想要旅行,
你想要高品质生活。你那么年轻却窥觑整个世界,你那么浮躁却想看透生活。你不断催促
自己赶快成长,却沉不下心来认真第一篇文章。你一次次吹响前进的号角,却总是倒在
离出发不远的地方。

\mfApache \quad 命是弱者的借口,运是强者的谦词。

\mfApache \quad 

\section{九州十色}
\mfApache \quad 酒入豪肠,七分酿成了月光,余下的三分啸成剑气,绣口一吐,就半个盛唐。

\mfApache \quad 月光还是少年的月光,九州一色还是李白的霜。

\mfApache \quad 月色与雪色之间,你是第三种绝色。

\mfApache \quad 愿中国青年都摆脱冷气,知识向上走,不必听自暴自弃者流的话。
能做事的做事,能发声的发声。
有一份热,发一分光,就令萤火一般,也可以在黑暗里发一点光,不必等候炬火。
伺候如竟没有炬火,我便是唯一的光。

\mfApache \quad 那时的我们有梦,关于文学,关于爱情,关于穿越世界的旅行
,如今我们深夜饮酒,杯子碰在一起,都是梦破碎的声音。

\mfApache \quad 她那时候还太年轻,不知道所有命运赠送的礼物,早已在暗中标好了价格。

\mfApache \quad 少年就是少年,他们看春风不喜,看夏蝉不烦,看秋风不悲,看冬雪不叹,
看满身富贵懒察觉,看不公不允敢面对,只因为他们是少年。

\mfApache \quad 我生怕自己本非美玉,故而不敢加以刻苦琢磨,却又半信自己十块美玉,
故又不肯庸庸碌碌,与瓦砾为伍。

\mfApache \quad 如果问我思念有多重,不重的,像一座秋山的落叶。

\mfApache \quad 世间情动,不过盛夏白瓷梅子汤,碎冰碰壁当啷响。

\mfApache \quad 

\section{青灯不灭}
\mfApache \quad 
无所谓从哪里来,也无所谓要到哪里去。为活着找个理由,只是为了更好的活着。

\chapter{古诗词选}
\section{诗无痕}
%-----------------------------------《诗无痕》
\mfApache \quad 一场秋雨一场凉,秋心酌满泪为霜。

\mfApache \quad 恰沐春风共同游,终只叹,木已舟。

\mfApache \quad 人间四月芳菲尽,山寺桃花始盛开。

\mfApache \quad 有三秋桂子,十里桃花。

\mfApache \quad 西风多少恨,吹不散眉弯。

\mfApache \quad 追风赶月莫停留,平芜尽处是春山。

\mfApache \quad 寄言燕雀莫相啅,自有云霄万里高。

\center{<-----那些浪漫温柔的官宣诗词----->}

\flushleft{\mfApache 金风玉露一相逢,便胜却人间无数。}

\flushright{——秦观《鹊桥仙·纤云弄巧》}

\flushleft{\mfApache 山似玉,玉如君,相看一笑温}

\flushright{——《更漏子·雪中韩叔夏席上》}

\flushleft{\mfApache 相怜相念倍相亲,一生一代一双人}

\flushright{——骆宾王《代女道士王灵妃赠道士李荣》}

\flushleft{\mfApache 愿为西南风,长逝入君怀}

\flushright{——曹植《明月上高楼》}

\flushleft{\mfApache 我欲与君相知,长命无绝衰。山无陵,江水为竭,冬雷震震,
夏雨雪,天地合,乃敢君绝。}

\flushright{《上邪》}

\flushleft{\mfApache 愿我如星君如月,夜夜流光相皎洁}

\flushright{——范成大《车遥遥篇》}

\flushleft{\mfApache 若似月轮终皎洁,不辞冰雪为卿热}

\flushright{——纳兰性德《蝶恋花·辛苦最怜天上月》}

\flushleft{\mfApache 只愿君心似我心,定不负相思意}

\flushright{——李之仪《卜算子》}

\flushleft{\mfApache 既见君子,云胡不喜}

\flushright{——《诗经·风雨》}

\flushleft{\mfApache 报我以木桃,报之以琼瑶。匪报也,永以为好也}

\flushright{——《诗经·木瓜》}

\section{言无尽}
\mfApache \quad \quad\quad\quad\quad \quad\quad\quad \quad\quad\textbf{滕王阁序}

豫章故郡,洪都新府。星分翼轸,地接衡庐。
襟三江而带五湖,控蛮荆而引瓯越。
物华天宝,龙光射牛斗之墟;人杰地灵,徐孺下陈蕃之榻。
雄州雾列,俊采星驰。台隍枕夷夏之交,宾主尽东南之美。
都督阎公之雅望,棨戟遥临;宇文新州之懿范,襜帷暂驻。
十旬休假,胜友如云;千里逢迎,高朋满座。
腾蛟起凤,孟学士之词宗;紫电青霜,王将军之武库。
家君作宰,路出名区;童子何知,躬逢胜饯。

时维九月,序属三秋。潦水尽而寒潭清,烟光凝而暮山紫。
俨骖騑于上路,访风景于崇阿。临帝子之长洲,得天人之旧馆。
层峦耸翠,上出重霄;飞阁流丹,下临无地。
鹤汀凫渚,穷岛屿之萦回;桂殿兰宫,即冈峦之体势。

披绣闼,俯雕甍,山原旷其盈视,川泽纡其骇瞩。
闾阎扑地,钟鸣鼎食之家;舸舰弥津,青雀黄龙之舳。
云销雨霁,彩彻区明。落霞与孤鹜齐飞,秋水共长天一色。
渔舟唱晚,响穷彭蠡之滨,雁阵惊寒,声断衡阳之浦。

遥襟甫畅,逸兴遄飞。爽籁发而清风生,纤歌凝而白云遏。
睢园绿竹,气凌彭泽之樽;邺水朱华,光照临川之笔。
四美具,二难并。穷睇眄于中天,极娱游于暇日。
天高地迥,觉宇宙之无穷;兴尽悲来,识盈虚之有数。
望长安于日下,目吴会于云间。地势极而南溟深,天柱高而北辰远。
关山难越,谁悲失路之人;萍水相逢,尽是他乡之客。
怀帝阍而不见,奉宣室以何年?

嗟乎!时运不齐,命途多舛。冯唐易老,李广难封。
屈贾谊于长沙,非无圣主;窜梁鸿于海曲,岂乏明时?
所赖君子见机,达人知命。老当益壮,宁移白首之心?穷且益坚,不坠青云之志。
酌贪泉而觉爽,处涸辙以犹欢。北海虽赊,扶摇可接;东隅已逝,桑榆非晚。
孟尝高洁,空余报国之情;阮籍猖狂,岂效穷途之哭!

勃,三尺微命,一介书生。无路请缨,等终军之弱冠;有怀投笔,慕宗悫之长风。
舍簪笏于百龄,奉晨昏于万里。
非谢家之宝树,接孟氏之芳邻。他日趋庭,叨陪鲤对;
今兹捧袂,喜托龙门。杨意不逢,抚凌云而自惜;钟期既遇,奏流水以何惭?

呜呼!胜地不常,盛筵难再;兰亭已矣,梓泽丘墟。临别赠言,幸承恩于伟饯;登高作赋,是所望于群公。敢竭鄙怀,恭疏短引;一言均赋,四韵俱成。请洒潘江,各倾陆海云尔:

\quad \quad 滕王高阁临江渚,佩玉鸣鸾罢歌舞。

\quad \quad 画栋朝飞南浦云,珠帘暮卷西山雨。

\quad \quad 闲云潭影日悠悠,物换星移几度秋。

\quad \quad 阁中帝子今何在?槛外长江空自流。

\mfApache \quad 人生海海,山山而川

\mfApache \quad 云月相同,溪山各异

\mfApache \quad 树犹如此,人何以堪。

\mfApache \quad 柔情似水,佳期如梦。

\mfApache \quad 但为君故,沉吟至今。

\mfApache \quad 生有热烈,藏于俗常。

\mfApache \quad 鹤别青山,不见桃花。

\mfApache \quad 知名不惧,日日自新。

\mfApache \quad 栀子横路,暗柳挟风。

\mfApache \quad 碧山人来,清酒深杯。

\mfApache \quad 所思不远,若为平生。

\mfApache \quad 海棠未雨,梨花先雪,一半春休。

\mfApache \quad 
%-----------------------------------《诗无痕》
\chapter{选段感想}

\chapter{咬文嚼字}
\section{名词大杂烩}
\mfApache \quad \underline{瓦子} \quad 瓦肆,也称瓦、瓦舍、瓦子、瓦市、三瓦两舍、三瓦两巷。
宋元时期大城市里茶楼、酒肆、剧场等娱乐场所的总称,是表演各种伎艺的综合性游艺场所。
“瓦”取“来时瓦合,去时瓦解”,野合易散,没阶级限制、来去自由之意

瓦肆兴起于宋仁宗和神宗时期。当时应市民娱乐之需要,原来分散在各地的各种民间伎艺汇集到城市内演出。内设围栏或者布帘划分表演区域和观众区域,称之为勾阑、勾栏。游艺项目很多,上演技艺达到百种,以说话、傀儡、杂技、杂剧、影戏最为盛行。瓦舍里开设经营服装、玩物、杂货、茶楼、酒肆、理发等店铺。瓦舍盛行于北宋,
南宋以后盛行于临安府,城内瓦舍归修内司管理,城外归殿前司。

\section{词语摘录}
\mfApache \quad \underline{行将就木} \quad 
指人寿命已经不长,快要进棺材了,比喻人临近死亡。

\mfApache \quad 

\section{杂项知识}

\end{document}